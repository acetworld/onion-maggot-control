\documentclass{article}
\usepackage[utf8]{inputenc}
\usepackage{changepage}% http://ctan.org/pkg/changepage
\usepackage{float}
\usepackage{fancyhdr}
\usepackage{lastpage}
\usepackage{graphicx}
\usepackage{ragged2e}
\usepackage{scrextend}
\usepackage{lastpage}
\usepackage{geometry}
 \geometry{
 left = 1.5in,
 right = 1.5in,
 top = 1in
 }
\pagestyle{fancy}
\renewcommand{\headrulewidth}{0pt}
\rhead{\includegraphics[height= 1cm]{ento1.png}\vspace{1em}}
\chead{\includegraphics[width = 5cm]{agritech1.png}}
\lhead{\includegraphics[width=3cm]{acet1.png}}
 
\fancyfoot{}
\setlength\headheight{40.8pt}
\RequirePackage[colorlinks=true, allcolors=blue]{hyperref}
\usepackage{titlesec}
\titleformat{\section}{\normalfont\fontsize{10}{15}\bfseries}{\thesection}{0em}{}
\titlespacing{\section}{}{1.24em}{0.24em}
 
\usepackage[default]{lato}
\usepackage[T1]{fontenc}

 
\cfoot{Page \thepage \hspace{1pt} of \pageref{LastPage}}

\begin{document}

\begin{addmargin}[3.8in]{}
Willett, Filgueiras, \newline
Nyrop and Nault \newline
----------------------------- \newline
Denis S. Willett\newline
Department of Entomology \newline
Cornell AgriTech  \newline
Cornell University \newline
15 Castle Creek Drive \newline
Geneva, NY 14456 \newline
deniswillett@cornell.edu \newline
\end{addmargin}
\setlength{\parindent}{0cm}

December 24, 2019

\vspace{1.24em}

Pest Management Science

\vspace{1.24em}

Dear Dr. Torto,

\vspace{0.48em}
\setlength{\parindent}{1.24cm}

Thank you for your consideration of \textit{“Attract and kill: Spinosad containing spheres to control onion maggot (\textit{Delia antiqua})”} for publication in Pest Management Science.  We have revised the manuscript in accordance with reviewer recommendations.  A complete and detailed account of the revisions is included below.  The reviewer comments were very helpful, much appreciated, and have resulted in a better manuscript.  In addition to submitting a revised version of the manuscript incorporating the latest changes, we are also submitting production quality images.  
  

\vspace{2em}

Much appreciated, \newline

\vspace{1em}

Denis S. Willett
\vspace{0.48em}

Camila C. Filgueiras
\vspace{0.48em}

Jan P. Nyrop
\vspace{0.48em}

Brian A. Nault
\vspace{0.48em} 


\vspace{1.4em}

\hline

\setlength\parindent{0pt}

\subsection*{Referee 1}

The authors did a great job writing the manuscript.  The introduction, methods, and results were easy to understand.    

\begin{quote}
    \textit{Thank you for reading the manuscript and providing constructive feedback.  We appreciate the suggestions and have endeavored to include them in the revised version.  }
\end{quote}

On lines 54 and 55 the authors state field trials were conducted from 2006 to 2008, but they did NOT describe if all experiments were repeated in 2006, 2007, and 2008.  The authors need to describe which experiments were conducted each year, how many times each experiment was replicated, and if the experiments were replicated in space and time.  

\begin{quote}
    \textit{Thanks for the opportunity to clarify.  Details on the timing of trials are now included in at the beginning of each sub-section in the Material and Methods section on lines ....  In short, the trials were replicated in space and time.  }
\end{quote}


The authors should add a few sentences describing the field sites (lines 54-56) cropping history (e.g. Where onions grown in previous years?) and the reason why the fields were chosen (e.g. did the fields have a previous history of onion maggot damage?).  If experiments were not replicated in space or time it should be noted in the Material and Methods. 

\begin{quote}
    \textit{Thanks for mentioning this opportunity to elaborate.  Details on cropping history and reasoning behind field site choice have now been included on lines ....  Information on how experiments were replicated in space and time are also included (see above).  }
\end{quote}

In the discussion section the authors made no mention of using spinosad seed treatment as a management option in conventional and organic onion fields.  Why?? Spinosad (Regard) seed treatment is currently the primary management method for onion maggot in organic onions, and spinosad is an active ingredient in many conventional seed treatments.   Spinosad is an active ingredient in the FI500 seed treatment, and Regard alone often provides a similar level of protection compared to FI500.  Since FI500 resulted in lower maggot damage compared to all "attract and kill" tactics, why would someone recommend attract and kill with spinosad over spinosad seed treatment?   I strongly recommend the authors discuss the advantages, disadvantages, and compatibility of using spinosad seed treatment and spinosad attract & kill.    




The authors should discuss the implications of onion maggot developing resistance to spinosad using the attract and kill tactic.  e.g. Will the attract and kill tactic with spinosad lead to a higher likelihood of onion maggot developing resistance to spinosad?  This tactic targets egg laying adults, it only provides partial control, and it exposes multiple generations per year to spinosad?   These factors seem to favor resistance. This question is especially relevant as the most popular registered insecticide seed treatments contain spinosad and there are few alternatives in the marketplace.     

\subsection*{Referee 2}

The methods and results need some significant improvements. I found that methods lacking in details and the analysis not well described. I also noted in the annotated file that the quality of the writing in the intro and discussion is markedly different and will need some work to bring them to the same level.

\begin{quote}
    \textit{Thank you for your comments regarding the manuscript.  We have added extended details to the methods section describing both the field work and analysis.  Additionally, we have adjusted language in the introduction and discussion.  We appreciate your feedback in helping improve the manuscript.  }
\end{quote}

Line 60: Please check and correct.  

\begin{quote}
    \textit{Thanks for the opportunity to verify this.  It is correct.  'Abut' (spelled exactly that way) means to share a common border. }
\end{quote}

Line 63: Not sure why its assumed that color and rate are related, explain.  

\begin{quote}
    \textit{Interesting question.  There is no assumption that color and rate are related.  The fact that white spheres are attractive to \textit{D. antiqua} is based on previous research in this system.  }
\end{quote}

Line 65: Does this sentence belong to the next paragraph? move it there.

\begin{quote}
    \textit{Excellent suggestion.  This sentence does seem to be hanging there.  Rather than include it in the subsequent paragraph, we have opted to remove it alltogether.  }
\end{quote}

Line 66: be more specific.  

\begin{quote}
    \textit{Thanks for the opportunity to elaborate.  We have included more details as to the timing and location of the trials throughout the methods section, specifically lines .... }
\end{quote}


Line 68: climate or weather?

\begin{quote}
    \textit{Interesting question.  We want to use the weather in these trials as a proxy for general exposure to northern climates and have revised the sentence in accordance with your suggestion.  }
\end{quote}


Line 68-69: This was already state above, seems redundant.

\begin{quote}
    \textit{Thanks for mentioning this.  We have removed the redundant statement.  }
\end{quote}

Line 71: More details are needed: how many reps/field, how many fields, how much distance between spheres, etc. Where in the field relative to the crop were spheres? What were they placed on and how high?

\begin{quote}
    \textit{Thanks for the opportunity to elaborate.  Details about reps, fields, distances, location, and height have been included on lines ....}
\end{quote}

Line 77: Which years was this experiment done? Throughout this section pls change to using SI units.

\begin{quote}
    \textit{Thanks for the suggestions.  Information on the timing of the experiment has been added to line.... Units have also been changed to SI.  Details have also been added regarding replications and field sites (lines ...).  }
\end{quote}

Line 91: was this based on a preliminary experiment?

\begin{quote}
    \textit{This was indeed based on previous work establishing the attractiveness of the spheres (Willett et al 2019). }
\end{quote}

Line 99: novel? How was it different from previously used ones?

\begin{quote}
    \textit{New but not novel.  It differed from previous formulations with the addition of the two extra components included below.  We have elaborated on the reasons for developing this on lines.   }
\end{quote}

Line 100: why were these compounds chosen?

\begin{quote}
    \textit{Excellent question.  The thought was that these compounds would help sterilize the paraffin mixture and enhance performance of the spheres over the long term.  }
\end{quote}

Line 104: was this true for previous experiments as well? If yes, please mention above.

\begin{quote}
    \textit{This was indeed true for previous experiments as well.  Thanks for the suggestion; these details have been included in the sections above as well.  }
\end{quote}

Line 108: which years?

\begin{quote}
    \textit{Great question.  This information has been included and elaborated upon in all methods sections on lines...}
\end{quote}

Line 110: pls explain how it was analyzed in addition to how it was displayed.

\begin{quote}
    \textit{Thanks for your inquiry.  It was analyzed using a smoothing function consisting of a generalized additive model with cubic splines.  }
\end{quote}

Line 112: aren't these the same thing?

\begin{quote}
    \textit{Thanks for your question.  Yes and No.  While linear models and analysis of variance use the same underlying model structure, they have two different objectives and answer two different questions.  Linear models represent the relationship between independent and dependent variables.  Analysis of variance compares inter and intra-group variation to examine differences.  In our case, both approaches were used because we are interested both in modeling relationships and differences between groups.  }
\end{quote}

Line 113: I'd like to get more details about which factors you have used, fixed and random.  

\begin{quote}
    \textit{}
\end{quote}

Line 115: AIC?, BIC? How did you use these to inform model selection? Considerations: third time this word is used in this sentence.  

Line 116:  I suggest rewording this sentence. I'm assuming that you are talking about means separation here?

Why use both, why not just one or the other? be more explicit about which model goes with what data. I'd like to see a list of dependent and independent variables and fixed and random effects.I suggest removing such a blanket statement and being more specific as I mentioned above.

Line 124: need to mention these where you talk about how you built and selected models. These are the tools you used to do that.

Line 128: I'm not following why showing this data is relevant. You also have not done analysis on it. Suggest incorporating it into the rest of the MS effectively.

Line 129: What period was this? Be more specific

Line 130: how were they sexed? Mention in the methods.  

Line 131: delete "roughly"

Line 133: What do you mean? Significantly different? Different from what?

\end{document}
